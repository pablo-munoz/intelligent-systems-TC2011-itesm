
% Default to the notebook output style

    


% Inherit from the specified cell style.




    
\documentclass[11pt]{article}

    
    
    \usepackage[T1]{fontenc}
    % Nicer default font (+ math font) than Computer Modern for most use cases
    \usepackage{mathpazo}

    % Basic figure setup, for now with no caption control since it's done
    % automatically by Pandoc (which extracts ![](path) syntax from Markdown).
    \usepackage{graphicx}
    % We will generate all images so they have a width \maxwidth. This means
    % that they will get their normal width if they fit onto the page, but
    % are scaled down if they would overflow the margins.
    \makeatletter
    \def\maxwidth{\ifdim\Gin@nat@width>\linewidth\linewidth
    \else\Gin@nat@width\fi}
    \makeatother
    \let\Oldincludegraphics\includegraphics
    % Set max figure width to be 80% of text width, for now hardcoded.
    \renewcommand{\includegraphics}[1]{\Oldincludegraphics[width=.8\maxwidth]{#1}}
    % Ensure that by default, figures have no caption (until we provide a
    % proper Figure object with a Caption API and a way to capture that
    % in the conversion process - todo).
    \usepackage{caption}
    \DeclareCaptionLabelFormat{nolabel}{}
    \captionsetup{labelformat=nolabel}

    \usepackage{adjustbox} % Used to constrain images to a maximum size 
    \usepackage{xcolor} % Allow colors to be defined
    \usepackage{enumerate} % Needed for markdown enumerations to work
    \usepackage{geometry} % Used to adjust the document margins
    \usepackage{amsmath} % Equations
    \usepackage{amssymb} % Equations
    \usepackage{textcomp} % defines textquotesingle
    % Hack from http://tex.stackexchange.com/a/47451/13684:
    \AtBeginDocument{%
        \def\PYZsq{\textquotesingle}% Upright quotes in Pygmentized code
    }
    \usepackage{upquote} % Upright quotes for verbatim code
    \usepackage{eurosym} % defines \euro
    \usepackage[mathletters]{ucs} % Extended unicode (utf-8) support
    \usepackage[utf8x]{inputenc} % Allow utf-8 characters in the tex document
    \usepackage{fancyvrb} % verbatim replacement that allows latex
    \usepackage{grffile} % extends the file name processing of package graphics 
                         % to support a larger range 
    % The hyperref package gives us a pdf with properly built
    % internal navigation ('pdf bookmarks' for the table of contents,
    % internal cross-reference links, web links for URLs, etc.)
    \usepackage{hyperref}
    \usepackage{longtable} % longtable support required by pandoc >1.10
    \usepackage{booktabs}  % table support for pandoc > 1.12.2
    \usepackage[inline]{enumitem} % IRkernel/repr support (it uses the enumerate* environment)
    \usepackage[normalem]{ulem} % ulem is needed to support strikethroughs (\sout)
                                % normalem makes italics be italics, not underlines
    

    
    
    % Colors for the hyperref package
    \definecolor{urlcolor}{rgb}{0,.145,.698}
    \definecolor{linkcolor}{rgb}{.71,0.21,0.01}
    \definecolor{citecolor}{rgb}{.12,.54,.11}

    % ANSI colors
    \definecolor{ansi-black}{HTML}{3E424D}
    \definecolor{ansi-black-intense}{HTML}{282C36}
    \definecolor{ansi-red}{HTML}{E75C58}
    \definecolor{ansi-red-intense}{HTML}{B22B31}
    \definecolor{ansi-green}{HTML}{00A250}
    \definecolor{ansi-green-intense}{HTML}{007427}
    \definecolor{ansi-yellow}{HTML}{DDB62B}
    \definecolor{ansi-yellow-intense}{HTML}{B27D12}
    \definecolor{ansi-blue}{HTML}{208FFB}
    \definecolor{ansi-blue-intense}{HTML}{0065CA}
    \definecolor{ansi-magenta}{HTML}{D160C4}
    \definecolor{ansi-magenta-intense}{HTML}{A03196}
    \definecolor{ansi-cyan}{HTML}{60C6C8}
    \definecolor{ansi-cyan-intense}{HTML}{258F8F}
    \definecolor{ansi-white}{HTML}{C5C1B4}
    \definecolor{ansi-white-intense}{HTML}{A1A6B2}

    % commands and environments needed by pandoc snippets
    % extracted from the output of `pandoc -s`
    \providecommand{\tightlist}{%
      \setlength{\itemsep}{0pt}\setlength{\parskip}{0pt}}
    \DefineVerbatimEnvironment{Highlighting}{Verbatim}{commandchars=\\\{\}}
    % Add ',fontsize=\small' for more characters per line
    \newenvironment{Shaded}{}{}
    \newcommand{\KeywordTok}[1]{\textcolor[rgb]{0.00,0.44,0.13}{\textbf{{#1}}}}
    \newcommand{\DataTypeTok}[1]{\textcolor[rgb]{0.56,0.13,0.00}{{#1}}}
    \newcommand{\DecValTok}[1]{\textcolor[rgb]{0.25,0.63,0.44}{{#1}}}
    \newcommand{\BaseNTok}[1]{\textcolor[rgb]{0.25,0.63,0.44}{{#1}}}
    \newcommand{\FloatTok}[1]{\textcolor[rgb]{0.25,0.63,0.44}{{#1}}}
    \newcommand{\CharTok}[1]{\textcolor[rgb]{0.25,0.44,0.63}{{#1}}}
    \newcommand{\StringTok}[1]{\textcolor[rgb]{0.25,0.44,0.63}{{#1}}}
    \newcommand{\CommentTok}[1]{\textcolor[rgb]{0.38,0.63,0.69}{\textit{{#1}}}}
    \newcommand{\OtherTok}[1]{\textcolor[rgb]{0.00,0.44,0.13}{{#1}}}
    \newcommand{\AlertTok}[1]{\textcolor[rgb]{1.00,0.00,0.00}{\textbf{{#1}}}}
    \newcommand{\FunctionTok}[1]{\textcolor[rgb]{0.02,0.16,0.49}{{#1}}}
    \newcommand{\RegionMarkerTok}[1]{{#1}}
    \newcommand{\ErrorTok}[1]{\textcolor[rgb]{1.00,0.00,0.00}{\textbf{{#1}}}}
    \newcommand{\NormalTok}[1]{{#1}}
    
    % Additional commands for more recent versions of Pandoc
    \newcommand{\ConstantTok}[1]{\textcolor[rgb]{0.53,0.00,0.00}{{#1}}}
    \newcommand{\SpecialCharTok}[1]{\textcolor[rgb]{0.25,0.44,0.63}{{#1}}}
    \newcommand{\VerbatimStringTok}[1]{\textcolor[rgb]{0.25,0.44,0.63}{{#1}}}
    \newcommand{\SpecialStringTok}[1]{\textcolor[rgb]{0.73,0.40,0.53}{{#1}}}
    \newcommand{\ImportTok}[1]{{#1}}
    \newcommand{\DocumentationTok}[1]{\textcolor[rgb]{0.73,0.13,0.13}{\textit{{#1}}}}
    \newcommand{\AnnotationTok}[1]{\textcolor[rgb]{0.38,0.63,0.69}{\textbf{\textit{{#1}}}}}
    \newcommand{\CommentVarTok}[1]{\textcolor[rgb]{0.38,0.63,0.69}{\textbf{\textit{{#1}}}}}
    \newcommand{\VariableTok}[1]{\textcolor[rgb]{0.10,0.09,0.49}{{#1}}}
    \newcommand{\ControlFlowTok}[1]{\textcolor[rgb]{0.00,0.44,0.13}{\textbf{{#1}}}}
    \newcommand{\OperatorTok}[1]{\textcolor[rgb]{0.40,0.40,0.40}{{#1}}}
    \newcommand{\BuiltInTok}[1]{{#1}}
    \newcommand{\ExtensionTok}[1]{{#1}}
    \newcommand{\PreprocessorTok}[1]{\textcolor[rgb]{0.74,0.48,0.00}{{#1}}}
    \newcommand{\AttributeTok}[1]{\textcolor[rgb]{0.49,0.56,0.16}{{#1}}}
    \newcommand{\InformationTok}[1]{\textcolor[rgb]{0.38,0.63,0.69}{\textbf{\textit{{#1}}}}}
    \newcommand{\WarningTok}[1]{\textcolor[rgb]{0.38,0.63,0.69}{\textbf{\textit{{#1}}}}}
    
    
    % Define a nice break command that doesn't care if a line doesn't already
    % exist.
    \def\br{\hspace*{\fill} \\* }
    % Math Jax compatability definitions
    \def\gt{>}
    \def\lt{<}
    % Document parameters
    \title{0380\_Assignment1\_BFS\_8puzzle}
    
    
    

    % Pygments definitions
    
\makeatletter
\def\PY@reset{\let\PY@it=\relax \let\PY@bf=\relax%
    \let\PY@ul=\relax \let\PY@tc=\relax%
    \let\PY@bc=\relax \let\PY@ff=\relax}
\def\PY@tok#1{\csname PY@tok@#1\endcsname}
\def\PY@toks#1+{\ifx\relax#1\empty\else%
    \PY@tok{#1}\expandafter\PY@toks\fi}
\def\PY@do#1{\PY@bc{\PY@tc{\PY@ul{%
    \PY@it{\PY@bf{\PY@ff{#1}}}}}}}
\def\PY#1#2{\PY@reset\PY@toks#1+\relax+\PY@do{#2}}

\expandafter\def\csname PY@tok@w\endcsname{\def\PY@tc##1{\textcolor[rgb]{0.73,0.73,0.73}{##1}}}
\expandafter\def\csname PY@tok@c\endcsname{\let\PY@it=\textit\def\PY@tc##1{\textcolor[rgb]{0.25,0.50,0.50}{##1}}}
\expandafter\def\csname PY@tok@cp\endcsname{\def\PY@tc##1{\textcolor[rgb]{0.74,0.48,0.00}{##1}}}
\expandafter\def\csname PY@tok@k\endcsname{\let\PY@bf=\textbf\def\PY@tc##1{\textcolor[rgb]{0.00,0.50,0.00}{##1}}}
\expandafter\def\csname PY@tok@kp\endcsname{\def\PY@tc##1{\textcolor[rgb]{0.00,0.50,0.00}{##1}}}
\expandafter\def\csname PY@tok@kt\endcsname{\def\PY@tc##1{\textcolor[rgb]{0.69,0.00,0.25}{##1}}}
\expandafter\def\csname PY@tok@o\endcsname{\def\PY@tc##1{\textcolor[rgb]{0.40,0.40,0.40}{##1}}}
\expandafter\def\csname PY@tok@ow\endcsname{\let\PY@bf=\textbf\def\PY@tc##1{\textcolor[rgb]{0.67,0.13,1.00}{##1}}}
\expandafter\def\csname PY@tok@nb\endcsname{\def\PY@tc##1{\textcolor[rgb]{0.00,0.50,0.00}{##1}}}
\expandafter\def\csname PY@tok@nf\endcsname{\def\PY@tc##1{\textcolor[rgb]{0.00,0.00,1.00}{##1}}}
\expandafter\def\csname PY@tok@nc\endcsname{\let\PY@bf=\textbf\def\PY@tc##1{\textcolor[rgb]{0.00,0.00,1.00}{##1}}}
\expandafter\def\csname PY@tok@nn\endcsname{\let\PY@bf=\textbf\def\PY@tc##1{\textcolor[rgb]{0.00,0.00,1.00}{##1}}}
\expandafter\def\csname PY@tok@ne\endcsname{\let\PY@bf=\textbf\def\PY@tc##1{\textcolor[rgb]{0.82,0.25,0.23}{##1}}}
\expandafter\def\csname PY@tok@nv\endcsname{\def\PY@tc##1{\textcolor[rgb]{0.10,0.09,0.49}{##1}}}
\expandafter\def\csname PY@tok@no\endcsname{\def\PY@tc##1{\textcolor[rgb]{0.53,0.00,0.00}{##1}}}
\expandafter\def\csname PY@tok@nl\endcsname{\def\PY@tc##1{\textcolor[rgb]{0.63,0.63,0.00}{##1}}}
\expandafter\def\csname PY@tok@ni\endcsname{\let\PY@bf=\textbf\def\PY@tc##1{\textcolor[rgb]{0.60,0.60,0.60}{##1}}}
\expandafter\def\csname PY@tok@na\endcsname{\def\PY@tc##1{\textcolor[rgb]{0.49,0.56,0.16}{##1}}}
\expandafter\def\csname PY@tok@nt\endcsname{\let\PY@bf=\textbf\def\PY@tc##1{\textcolor[rgb]{0.00,0.50,0.00}{##1}}}
\expandafter\def\csname PY@tok@nd\endcsname{\def\PY@tc##1{\textcolor[rgb]{0.67,0.13,1.00}{##1}}}
\expandafter\def\csname PY@tok@s\endcsname{\def\PY@tc##1{\textcolor[rgb]{0.73,0.13,0.13}{##1}}}
\expandafter\def\csname PY@tok@sd\endcsname{\let\PY@it=\textit\def\PY@tc##1{\textcolor[rgb]{0.73,0.13,0.13}{##1}}}
\expandafter\def\csname PY@tok@si\endcsname{\let\PY@bf=\textbf\def\PY@tc##1{\textcolor[rgb]{0.73,0.40,0.53}{##1}}}
\expandafter\def\csname PY@tok@se\endcsname{\let\PY@bf=\textbf\def\PY@tc##1{\textcolor[rgb]{0.73,0.40,0.13}{##1}}}
\expandafter\def\csname PY@tok@sr\endcsname{\def\PY@tc##1{\textcolor[rgb]{0.73,0.40,0.53}{##1}}}
\expandafter\def\csname PY@tok@ss\endcsname{\def\PY@tc##1{\textcolor[rgb]{0.10,0.09,0.49}{##1}}}
\expandafter\def\csname PY@tok@sx\endcsname{\def\PY@tc##1{\textcolor[rgb]{0.00,0.50,0.00}{##1}}}
\expandafter\def\csname PY@tok@m\endcsname{\def\PY@tc##1{\textcolor[rgb]{0.40,0.40,0.40}{##1}}}
\expandafter\def\csname PY@tok@gh\endcsname{\let\PY@bf=\textbf\def\PY@tc##1{\textcolor[rgb]{0.00,0.00,0.50}{##1}}}
\expandafter\def\csname PY@tok@gu\endcsname{\let\PY@bf=\textbf\def\PY@tc##1{\textcolor[rgb]{0.50,0.00,0.50}{##1}}}
\expandafter\def\csname PY@tok@gd\endcsname{\def\PY@tc##1{\textcolor[rgb]{0.63,0.00,0.00}{##1}}}
\expandafter\def\csname PY@tok@gi\endcsname{\def\PY@tc##1{\textcolor[rgb]{0.00,0.63,0.00}{##1}}}
\expandafter\def\csname PY@tok@gr\endcsname{\def\PY@tc##1{\textcolor[rgb]{1.00,0.00,0.00}{##1}}}
\expandafter\def\csname PY@tok@ge\endcsname{\let\PY@it=\textit}
\expandafter\def\csname PY@tok@gs\endcsname{\let\PY@bf=\textbf}
\expandafter\def\csname PY@tok@gp\endcsname{\let\PY@bf=\textbf\def\PY@tc##1{\textcolor[rgb]{0.00,0.00,0.50}{##1}}}
\expandafter\def\csname PY@tok@go\endcsname{\def\PY@tc##1{\textcolor[rgb]{0.53,0.53,0.53}{##1}}}
\expandafter\def\csname PY@tok@gt\endcsname{\def\PY@tc##1{\textcolor[rgb]{0.00,0.27,0.87}{##1}}}
\expandafter\def\csname PY@tok@err\endcsname{\def\PY@bc##1{\setlength{\fboxsep}{0pt}\fcolorbox[rgb]{1.00,0.00,0.00}{1,1,1}{\strut ##1}}}
\expandafter\def\csname PY@tok@kc\endcsname{\let\PY@bf=\textbf\def\PY@tc##1{\textcolor[rgb]{0.00,0.50,0.00}{##1}}}
\expandafter\def\csname PY@tok@kd\endcsname{\let\PY@bf=\textbf\def\PY@tc##1{\textcolor[rgb]{0.00,0.50,0.00}{##1}}}
\expandafter\def\csname PY@tok@kn\endcsname{\let\PY@bf=\textbf\def\PY@tc##1{\textcolor[rgb]{0.00,0.50,0.00}{##1}}}
\expandafter\def\csname PY@tok@kr\endcsname{\let\PY@bf=\textbf\def\PY@tc##1{\textcolor[rgb]{0.00,0.50,0.00}{##1}}}
\expandafter\def\csname PY@tok@bp\endcsname{\def\PY@tc##1{\textcolor[rgb]{0.00,0.50,0.00}{##1}}}
\expandafter\def\csname PY@tok@fm\endcsname{\def\PY@tc##1{\textcolor[rgb]{0.00,0.00,1.00}{##1}}}
\expandafter\def\csname PY@tok@vc\endcsname{\def\PY@tc##1{\textcolor[rgb]{0.10,0.09,0.49}{##1}}}
\expandafter\def\csname PY@tok@vg\endcsname{\def\PY@tc##1{\textcolor[rgb]{0.10,0.09,0.49}{##1}}}
\expandafter\def\csname PY@tok@vi\endcsname{\def\PY@tc##1{\textcolor[rgb]{0.10,0.09,0.49}{##1}}}
\expandafter\def\csname PY@tok@vm\endcsname{\def\PY@tc##1{\textcolor[rgb]{0.10,0.09,0.49}{##1}}}
\expandafter\def\csname PY@tok@sa\endcsname{\def\PY@tc##1{\textcolor[rgb]{0.73,0.13,0.13}{##1}}}
\expandafter\def\csname PY@tok@sb\endcsname{\def\PY@tc##1{\textcolor[rgb]{0.73,0.13,0.13}{##1}}}
\expandafter\def\csname PY@tok@sc\endcsname{\def\PY@tc##1{\textcolor[rgb]{0.73,0.13,0.13}{##1}}}
\expandafter\def\csname PY@tok@dl\endcsname{\def\PY@tc##1{\textcolor[rgb]{0.73,0.13,0.13}{##1}}}
\expandafter\def\csname PY@tok@s2\endcsname{\def\PY@tc##1{\textcolor[rgb]{0.73,0.13,0.13}{##1}}}
\expandafter\def\csname PY@tok@sh\endcsname{\def\PY@tc##1{\textcolor[rgb]{0.73,0.13,0.13}{##1}}}
\expandafter\def\csname PY@tok@s1\endcsname{\def\PY@tc##1{\textcolor[rgb]{0.73,0.13,0.13}{##1}}}
\expandafter\def\csname PY@tok@mb\endcsname{\def\PY@tc##1{\textcolor[rgb]{0.40,0.40,0.40}{##1}}}
\expandafter\def\csname PY@tok@mf\endcsname{\def\PY@tc##1{\textcolor[rgb]{0.40,0.40,0.40}{##1}}}
\expandafter\def\csname PY@tok@mh\endcsname{\def\PY@tc##1{\textcolor[rgb]{0.40,0.40,0.40}{##1}}}
\expandafter\def\csname PY@tok@mi\endcsname{\def\PY@tc##1{\textcolor[rgb]{0.40,0.40,0.40}{##1}}}
\expandafter\def\csname PY@tok@il\endcsname{\def\PY@tc##1{\textcolor[rgb]{0.40,0.40,0.40}{##1}}}
\expandafter\def\csname PY@tok@mo\endcsname{\def\PY@tc##1{\textcolor[rgb]{0.40,0.40,0.40}{##1}}}
\expandafter\def\csname PY@tok@ch\endcsname{\let\PY@it=\textit\def\PY@tc##1{\textcolor[rgb]{0.25,0.50,0.50}{##1}}}
\expandafter\def\csname PY@tok@cm\endcsname{\let\PY@it=\textit\def\PY@tc##1{\textcolor[rgb]{0.25,0.50,0.50}{##1}}}
\expandafter\def\csname PY@tok@cpf\endcsname{\let\PY@it=\textit\def\PY@tc##1{\textcolor[rgb]{0.25,0.50,0.50}{##1}}}
\expandafter\def\csname PY@tok@c1\endcsname{\let\PY@it=\textit\def\PY@tc##1{\textcolor[rgb]{0.25,0.50,0.50}{##1}}}
\expandafter\def\csname PY@tok@cs\endcsname{\let\PY@it=\textit\def\PY@tc##1{\textcolor[rgb]{0.25,0.50,0.50}{##1}}}

\def\PYZbs{\char`\\}
\def\PYZus{\char`\_}
\def\PYZob{\char`\{}
\def\PYZcb{\char`\}}
\def\PYZca{\char`\^}
\def\PYZam{\char`\&}
\def\PYZlt{\char`\<}
\def\PYZgt{\char`\>}
\def\PYZsh{\char`\#}
\def\PYZpc{\char`\%}
\def\PYZdl{\char`\$}
\def\PYZhy{\char`\-}
\def\PYZsq{\char`\'}
\def\PYZdq{\char`\"}
\def\PYZti{\char`\~}
% for compatibility with earlier versions
\def\PYZat{@}
\def\PYZlb{[}
\def\PYZrb{]}
\makeatother


    % Exact colors from NB
    \definecolor{incolor}{rgb}{0.0, 0.0, 0.5}
    \definecolor{outcolor}{rgb}{0.545, 0.0, 0.0}



    
    % Prevent overflowing lines due to hard-to-break entities
    \sloppy 
    % Setup hyperref package
    \hypersetup{
      breaklinks=true,  % so long urls are correctly broken across lines
      colorlinks=true,
      urlcolor=urlcolor,
      linkcolor=linkcolor,
      citecolor=citecolor,
      }
    % Slightly bigger margins than the latex defaults
    
    \geometry{verbose,tmargin=1in,bmargin=1in,lmargin=1in,rmargin=1in}
    
    

    \begin{document}
    
    
    \maketitle
    
    

    
    \hypertarget{puzzle-problem-solution-by-bfs}{%
\section{8 puzzle problem solution by
BFS}\label{puzzle-problem-solution-by-bfs}}

\begin{itemize}
\tightlist
\item
  Student Name: Pablo Muñoz Haro
\item
  Student ID: A01222422
\item
  Course name: TC2011 INTELLIGENT SYSTEMS
\item
  Date: August 12th, 2018
\item
  Assignment Name: UnInformed search -- BFS
\end{itemize}

The 8 puzzle problem is defined as follows: Consider a 3x3 (nine cell)
grid. One of the cells is ``blank'' and the rest contains the integers 1
through 8 without repetitions. Obvisouly the grid has the numbers and
the blank ordered in some initial arrangement. It is desired to move the
``cells'' around until the grid is in another, target arrangement (say
with the blank first and then the numbers in increasing order). You can
only move the cells adjacent to the blank and only to the blank's
position (think of it as you can swap the blank and one of its neighbors
positions). The goal is to come up with a move sequence (left, right, up
down) that will take you from the initial configuration to the target
configuration.

The solution by BFS consists of trying all possible moves at each stage,
for this problem, the state of the grid is encoded as a 9-tuple filled
with integers. We use a 0 to denote the ``blank'' and the numbers 1
through 8 represent themselves. For example, the state (1, 3, 5, 7, 0,
8, 2, 4, 6) represents the grid:

\begin{verbatim}
| 1 | 3 | 5 |
| 7 |   | 8 |
| 2 | 4 | 6 |
\end{verbatim}

Tuples are used to store the grid state due to their ability to be
hashable, which allows us to use them in sets and test for membership
easily. For the queue we use python's standard \texttt{deque} which
allow for very efficient additions and removals both at the start and at
the end of the collection.

The majority of the work is done by the function
\texttt{eight\_puzzle\_breadth\_first\_search(initial\_state,\ final\_state)}.
This function mantains a queue (deque) that starts with the initial
state. Then, while the queue is not empty and the goal hasn't been
reached an element is dequeued from the queue, and every possible
neighbor is calculated and put back into the queue (a neighbor is a
state that would result from doing an UP, DOWN, LEFT or RIGHT, move).
\texttt{eight\_puzzle\_breadth\_first\_search} implements an inner
function \texttt{swap} that just makes it easy to swap two elements in a
list.

To use this notebook simply modify the FINAL\_STATE and INITIAL\_STATE
variables tha appear a few cells below.

    \begin{Verbatim}[commandchars=\\\{\}]
{\color{incolor}In [{\color{incolor}1}]:} \PY{k+kn}{from} \PY{n+nn}{collections} \PY{k}{import} \PY{n}{deque}\PY{p}{,} \PY{n}{namedtuple}
        \PY{k+kn}{import} \PY{n+nn}{time}
\end{Verbatim}


    \begin{Verbatim}[commandchars=\\\{\}]
{\color{incolor}In [{\color{incolor}2}]:} \PY{n}{eight\PYZus{}puzzle\PYZus{}solution} \PY{o}{=} \PY{n}{namedtuple}\PY{p}{(}\PY{l+s+s1}{\PYZsq{}}\PY{l+s+s1}{eight\PYZus{}puzzle\PYZus{}solution}\PY{l+s+s1}{\PYZsq{}}\PY{p}{,} \PY{p}{[}
            \PY{l+s+s1}{\PYZsq{}}\PY{l+s+s1}{path\PYZus{}to\PYZus{}goal}\PY{l+s+s1}{\PYZsq{}}\PY{p}{,} \PY{l+s+s1}{\PYZsq{}}\PY{l+s+s1}{path\PYZus{}cost}\PY{l+s+s1}{\PYZsq{}}\PY{p}{,} \PY{l+s+s1}{\PYZsq{}}\PY{l+s+s1}{num\PYZus{}visited\PYZus{}nodes}\PY{l+s+s1}{\PYZsq{}}\PY{p}{,}
            \PY{l+s+s1}{\PYZsq{}}\PY{l+s+s1}{running\PYZus{}time\PYZus{}secs}\PY{l+s+s1}{\PYZsq{}}\PY{p}{,} \PY{l+s+s1}{\PYZsq{}}\PY{l+s+s1}{used\PYZus{}memory}\PY{l+s+s1}{\PYZsq{}}
        \PY{p}{]}\PY{p}{)}
\end{Verbatim}


    \begin{Verbatim}[commandchars=\\\{\}]
{\color{incolor}In [{\color{incolor}3}]:} \PY{n}{FINAL\PYZus{}STATE} \PY{o}{=} \PY{p}{(}\PY{l+m+mi}{0}\PY{p}{,} \PY{l+m+mi}{1}\PY{p}{,} \PY{l+m+mi}{2}\PY{p}{,} \PY{l+m+mi}{3}\PY{p}{,} \PY{l+m+mi}{4}\PY{p}{,} \PY{l+m+mi}{5}\PY{p}{,} \PY{l+m+mi}{6}\PY{p}{,} \PY{l+m+mi}{7}\PY{p}{,} \PY{l+m+mi}{8}\PY{p}{)}
\end{Verbatim}


    \begin{Verbatim}[commandchars=\\\{\}]
{\color{incolor}In [{\color{incolor}4}]:} \PY{n}{INITIAL\PYZus{}STATE} \PY{o}{=} \PY{p}{(}\PY{l+m+mi}{7}\PY{p}{,} \PY{l+m+mi}{2}\PY{p}{,} \PY{l+m+mi}{4}\PY{p}{,} \PY{l+m+mi}{5}\PY{p}{,} \PY{l+m+mi}{0}\PY{p}{,} \PY{l+m+mi}{6}\PY{p}{,} \PY{l+m+mi}{8}\PY{p}{,} \PY{l+m+mi}{3}\PY{p}{,} \PY{l+m+mi}{1}\PY{p}{)}
\end{Verbatim}


    \begin{Verbatim}[commandchars=\\\{\}]
{\color{incolor}In [{\color{incolor}5}]:} \PY{k}{def} \PY{n+nf}{eight\PYZus{}puzzle\PYZus{}breadth\PYZus{}first\PYZus{}search}\PY{p}{(}\PY{n}{initial\PYZus{}state}\PY{p}{,} \PY{n}{final\PYZus{}state}\PY{p}{)}\PY{p}{:}
            \PY{l+s+sd}{\PYZsq{}\PYZsq{}\PYZsq{}}
        \PY{l+s+sd}{    Solves the eight puzzle problem through the breadth first}
        \PY{l+s+sd}{    search algorithm.}
        \PY{l+s+sd}{    }
        \PY{l+s+sd}{    Args:}
        \PY{l+s+sd}{        initial\PYZus{}state (tuple): A tuple representing the initial}
        \PY{l+s+sd}{            state of the problem. It is assumed that the tuple is}
        \PY{l+s+sd}{            of length 9 and contains the numbers 0 through 8 without}
        \PY{l+s+sd}{            repetitions.}
        \PY{l+s+sd}{        final\PYZus{}state (tuple): A tuple representing the desired final}
        \PY{l+s+sd}{            state of the problem. It is assumed that the tuple is}
        \PY{l+s+sd}{            of length 9 and contains the numbers 0 through 8 without}
        \PY{l+s+sd}{            repetitions.}
        \PY{l+s+sd}{          }
        \PY{l+s+sd}{    Returns:}
        \PY{l+s+sd}{        eight\PYZus{}puzzle\PYZus{}solution namedtuple: A number of statistics}
        \PY{l+s+sd}{            about the solution and the effor it took to achieve it.}
        \PY{l+s+sd}{    \PYZsq{}\PYZsq{}\PYZsq{}}
            \PY{n}{BRANCHING\PYZus{}FACTOR} \PY{o}{=} \PY{l+m+mi}{9}
            \PY{n}{UNIT\PYZus{}COST} \PY{o}{=} \PY{l+m+mi}{72}
            
            \PY{k}{def} \PY{n+nf}{swap}\PY{p}{(}\PY{n}{container}\PY{p}{,} \PY{n}{index\PYZus{}1}\PY{p}{,} \PY{n}{index\PYZus{}2}\PY{p}{)}\PY{p}{:}
                \PY{l+s+sd}{\PYZsq{}\PYZsq{}\PYZsq{}}
        \PY{l+s+sd}{        Utility function to swap two elements of a collection.}
        \PY{l+s+sd}{        Used to calculate the next state after moving the 8\PYZhy{}puzzle}
        \PY{l+s+sd}{        pivot one spot to the left, right, up or down.}
        \PY{l+s+sd}{        }
        \PY{l+s+sd}{        Args:}
        \PY{l+s+sd}{            container (list): A mutable representation of a state}
        \PY{l+s+sd}{            index\PYZus{}1 (int): One of the indexes to swap}
        \PY{l+s+sd}{            index\PYZus{}2 (int): The other index to swap}
        \PY{l+s+sd}{            }
        \PY{l+s+sd}{        Returns:}
        \PY{l+s+sd}{            None}
        \PY{l+s+sd}{            }
        \PY{l+s+sd}{        Side\PYZhy{}Effects:}
        \PY{l+s+sd}{            Mutates container so that elements at positions}
        \PY{l+s+sd}{            index\PYZus{}1 and index\PYZus{}2 are swapped}
        \PY{l+s+sd}{        \PYZsq{}\PYZsq{}\PYZsq{}}
                \PY{n}{container}\PY{p}{[}\PY{n}{index\PYZus{}1}\PY{p}{]}\PY{p}{,} \PY{n}{container}\PY{p}{[}\PY{n}{index\PYZus{}2}\PY{p}{]} \PY{o}{=} \PY{n}{container}\PY{p}{[}\PY{n}{index\PYZus{}2}\PY{p}{]}\PY{p}{,} \PY{n}{container}\PY{p}{[}\PY{n}{index\PYZus{}1}\PY{p}{]}
            
            \PY{n}{frontier} \PY{o}{=} \PY{n}{deque}\PY{p}{(}\PY{p}{(}\PY{n}{initial\PYZus{}state}\PY{p}{,}\PY{p}{)}\PY{p}{)}
            \PY{n}{visited} \PY{o}{=} \PY{n+nb}{set}\PY{p}{(}\PY{p}{)}
            \PY{n}{start\PYZus{}time} \PY{o}{=} \PY{n}{time}\PY{o}{.}\PY{n}{time}\PY{p}{(}\PY{p}{)}
            \PY{n}{max\PYZus{}queue\PYZus{}size} \PY{o}{=} \PY{l+m+mi}{1} \PY{c+c1}{\PYZsh{} Mantain a count of the most number of nodes held in memory at any point}
            
            \PY{n}{move\PYZus{}sequences} \PY{o}{=} \PY{p}{\PYZob{}}
                \PY{n}{INITIAL\PYZus{}STATE}\PY{p}{:} \PY{l+s+s1}{\PYZsq{}}\PY{l+s+s1}{\PYZsq{}}
            \PY{p}{\PYZcb{}}
            
            \PY{k}{while} \PY{n+nb}{len}\PY{p}{(}\PY{n}{frontier}\PY{p}{)} \PY{o+ow}{is} \PY{o+ow}{not} \PY{l+m+mi}{0}\PY{p}{:}
                \PY{k}{if} \PY{n+nb}{len}\PY{p}{(}\PY{n}{frontier}\PY{p}{)} \PY{o}{\PYZgt{}} \PY{n}{max\PYZus{}queue\PYZus{}size}\PY{p}{:}
                    \PY{n}{max\PYZus{}queue\PYZus{}size} \PY{o}{=} \PY{n+nb}{len}\PY{p}{(}\PY{n}{frontier}\PY{p}{)}
                
                \PY{n}{node\PYZus{}to\PYZus{}expand} \PY{o}{=} \PY{n}{frontier}\PY{o}{.}\PY{n}{popleft}\PY{p}{(}\PY{p}{)}
                \PY{n}{visited} \PY{o}{|}\PY{o}{=} \PY{p}{\PYZob{}}\PY{n}{node\PYZus{}to\PYZus{}expand}\PY{p}{\PYZcb{}}
                
                \PY{k}{if} \PY{n}{node\PYZus{}to\PYZus{}expand} \PY{o}{==} \PY{n}{FINAL\PYZus{}STATE}\PY{p}{:}
                    \PY{n}{end\PYZus{}time} \PY{o}{=} \PY{n}{time}\PY{o}{.}\PY{n}{time}\PY{p}{(}\PY{p}{)}
                    \PY{n}{elapsed\PYZus{}time} \PY{o}{=} \PY{n}{end\PYZus{}time} \PY{o}{\PYZhy{}} \PY{n}{start\PYZus{}time}
                    \PY{n}{path\PYZus{}to\PYZus{}goal} \PY{o}{=} \PY{n}{move\PYZus{}sequences}\PY{p}{[}\PY{n}{FINAL\PYZus{}STATE}\PY{p}{]}
                    \PY{n}{path\PYZus{}cost} \PY{o}{=} \PY{n+nb}{len}\PY{p}{(}\PY{n}{path\PYZus{}to\PYZus{}goal}\PY{p}{)}
                    \PY{n}{num\PYZus{}visited\PYZus{}nodes} \PY{o}{=} \PY{n+nb}{len}\PY{p}{(}\PY{n}{visited}\PY{p}{)}
                    \PY{n}{used\PYZus{}memory} \PY{o}{=} \PY{n}{max\PYZus{}queue\PYZus{}size}\PY{o}{*}\PY{n}{UNIT\PYZus{}COST}
                    
                    \PY{k}{return} \PY{n}{eight\PYZus{}puzzle\PYZus{}solution}\PY{p}{(}
                        \PY{n}{path\PYZus{}to\PYZus{}goal}\PY{p}{,}
                        \PY{n}{path\PYZus{}cost}\PY{p}{,}
                        \PY{n}{num\PYZus{}visited\PYZus{}nodes}\PY{p}{,}
                        \PY{n}{elapsed\PYZus{}time}\PY{p}{,}
                        \PY{n}{used\PYZus{}memory}\PY{p}{,}
                    \PY{p}{)}
                
                \PY{n}{pivot\PYZus{}index} \PY{o}{=} \PY{n}{node\PYZus{}to\PYZus{}expand}\PY{o}{.}\PY{n}{index}\PY{p}{(}\PY{l+m+mi}{0}\PY{p}{)}
                \PY{n}{neighbors} \PY{o}{=} \PY{p}{[}\PY{p}{]}
                
                \PY{n}{can\PYZus{}move\PYZus{}left} \PY{o}{=} \PY{n}{pivot\PYZus{}index} \PY{o+ow}{not} \PY{o+ow}{in} \PY{p}{(}\PY{l+m+mi}{0}\PY{p}{,} \PY{l+m+mi}{3}\PY{p}{,} \PY{l+m+mi}{6}\PY{p}{)}
                \PY{n}{can\PYZus{}move\PYZus{}right} \PY{o}{=} \PY{n}{pivot\PYZus{}index} \PY{o+ow}{not} \PY{o+ow}{in} \PY{p}{(}\PY{l+m+mi}{2}\PY{p}{,} \PY{l+m+mi}{5}\PY{p}{,} \PY{l+m+mi}{8}\PY{p}{)}
                \PY{n}{can\PYZus{}move\PYZus{}up} \PY{o}{=} \PY{n}{pivot\PYZus{}index} \PY{o+ow}{not} \PY{o+ow}{in} \PY{p}{(}\PY{l+m+mi}{0}\PY{p}{,} \PY{l+m+mi}{1}\PY{p}{,} \PY{l+m+mi}{2}\PY{p}{)}
                \PY{n}{can\PYZus{}move\PYZus{}down} \PY{o}{=} \PY{n}{pivot\PYZus{}index} \PY{o+ow}{not} \PY{o+ow}{in} \PY{p}{(}\PY{l+m+mi}{6}\PY{p}{,} \PY{l+m+mi}{7}\PY{p}{,} \PY{l+m+mi}{8}\PY{p}{)}
                
                \PY{k}{if} \PY{n}{can\PYZus{}move\PYZus{}left}\PY{p}{:}
                    \PY{n}{left\PYZus{}neighbor\PYZus{}listform} \PY{o}{=} \PY{n+nb}{list}\PY{p}{(}\PY{n}{node\PYZus{}to\PYZus{}expand}\PY{p}{)}
                    \PY{n}{swap}\PY{p}{(}\PY{n}{left\PYZus{}neighbor\PYZus{}listform}\PY{p}{,} \PY{n}{pivot\PYZus{}index}\PY{p}{,} \PY{n}{pivot\PYZus{}index} \PY{o}{\PYZhy{}} \PY{l+m+mi}{1}\PY{p}{)}
                    \PY{n}{left\PYZus{}neighbor} \PY{o}{=} \PY{n+nb}{tuple}\PY{p}{(}\PY{n}{left\PYZus{}neighbor\PYZus{}listform}\PY{p}{)}
                    \PY{k}{if} \PY{n}{left\PYZus{}neighbor} \PY{o+ow}{not} \PY{o+ow}{in} \PY{n}{move\PYZus{}sequences}\PY{p}{:}
                        \PY{n}{move\PYZus{}sequences}\PY{p}{[}\PY{n}{left\PYZus{}neighbor}\PY{p}{]} \PY{o}{=} \PY{n}{move\PYZus{}sequences}\PY{p}{[}\PY{n}{node\PYZus{}to\PYZus{}expand}\PY{p}{]} \PY{o}{+} \PY{l+s+s1}{\PYZsq{}}\PY{l+s+s1}{L}\PY{l+s+s1}{\PYZsq{}}
                    \PY{n}{neighbors}\PY{o}{.}\PY{n}{append}\PY{p}{(}\PY{n}{left\PYZus{}neighbor}\PY{p}{)}
                    
                \PY{k}{if} \PY{n}{can\PYZus{}move\PYZus{}right}\PY{p}{:}
                    \PY{n}{right\PYZus{}neighbor\PYZus{}listform} \PY{o}{=} \PY{n+nb}{list}\PY{p}{(}\PY{n}{node\PYZus{}to\PYZus{}expand}\PY{p}{)}
                    \PY{n}{swap}\PY{p}{(}\PY{n}{right\PYZus{}neighbor\PYZus{}listform}\PY{p}{,} \PY{n}{pivot\PYZus{}index}\PY{p}{,} \PY{n}{pivot\PYZus{}index} \PY{o}{+} \PY{l+m+mi}{1}\PY{p}{)}
                    \PY{n}{right\PYZus{}neighbor} \PY{o}{=} \PY{n+nb}{tuple}\PY{p}{(}\PY{n}{right\PYZus{}neighbor\PYZus{}listform}\PY{p}{)}
                    \PY{k}{if} \PY{n}{right\PYZus{}neighbor} \PY{o+ow}{not} \PY{o+ow}{in} \PY{n}{move\PYZus{}sequences}\PY{p}{:}
                        \PY{n}{move\PYZus{}sequences}\PY{p}{[}\PY{n}{right\PYZus{}neighbor}\PY{p}{]} \PY{o}{=} \PY{n}{move\PYZus{}sequences}\PY{p}{[}\PY{n}{node\PYZus{}to\PYZus{}expand}\PY{p}{]} \PY{o}{+} \PY{l+s+s1}{\PYZsq{}}\PY{l+s+s1}{R}\PY{l+s+s1}{\PYZsq{}}
                    \PY{n}{neighbors}\PY{o}{.}\PY{n}{append}\PY{p}{(}\PY{n}{right\PYZus{}neighbor}\PY{p}{)}
        
                \PY{k}{if} \PY{n}{can\PYZus{}move\PYZus{}up}\PY{p}{:}
                    \PY{n}{up\PYZus{}neighbor\PYZus{}listform} \PY{o}{=} \PY{n+nb}{list}\PY{p}{(}\PY{n}{node\PYZus{}to\PYZus{}expand}\PY{p}{)}
                    \PY{n}{swap}\PY{p}{(}\PY{n}{up\PYZus{}neighbor\PYZus{}listform}\PY{p}{,} \PY{n}{pivot\PYZus{}index}\PY{p}{,} \PY{n}{pivot\PYZus{}index} \PY{o}{\PYZhy{}} \PY{l+m+mi}{3}\PY{p}{)}
                    \PY{n}{up\PYZus{}neighbor} \PY{o}{=} \PY{n+nb}{tuple}\PY{p}{(}\PY{n}{up\PYZus{}neighbor\PYZus{}listform}\PY{p}{)}
                    \PY{k}{if} \PY{n}{up\PYZus{}neighbor} \PY{o+ow}{not} \PY{o+ow}{in} \PY{n}{move\PYZus{}sequences}\PY{p}{:}
                        \PY{n}{move\PYZus{}sequences}\PY{p}{[}\PY{n}{up\PYZus{}neighbor}\PY{p}{]} \PY{o}{=} \PY{n}{move\PYZus{}sequences}\PY{p}{[}\PY{n}{node\PYZus{}to\PYZus{}expand}\PY{p}{]} \PY{o}{+} \PY{l+s+s1}{\PYZsq{}}\PY{l+s+s1}{U}\PY{l+s+s1}{\PYZsq{}}
                    \PY{n}{neighbors}\PY{o}{.}\PY{n}{append}\PY{p}{(}\PY{n}{up\PYZus{}neighbor}\PY{p}{)}
        
                \PY{k}{if} \PY{n}{can\PYZus{}move\PYZus{}down}\PY{p}{:}
                    \PY{n}{down\PYZus{}neighbor\PYZus{}listform} \PY{o}{=} \PY{n+nb}{list}\PY{p}{(}\PY{n}{node\PYZus{}to\PYZus{}expand}\PY{p}{)}
                    \PY{n}{swap}\PY{p}{(}\PY{n}{down\PYZus{}neighbor\PYZus{}listform}\PY{p}{,} \PY{n}{pivot\PYZus{}index}\PY{p}{,} \PY{n}{pivot\PYZus{}index} \PY{o}{+} \PY{l+m+mi}{3}\PY{p}{)}
                    \PY{n}{down\PYZus{}neighbor} \PY{o}{=} \PY{n+nb}{tuple}\PY{p}{(}\PY{n}{down\PYZus{}neighbor\PYZus{}listform}\PY{p}{)}
                    \PY{k}{if} \PY{n}{down\PYZus{}neighbor} \PY{o+ow}{not} \PY{o+ow}{in} \PY{n}{move\PYZus{}sequences}\PY{p}{:}
                        \PY{n}{move\PYZus{}sequences}\PY{p}{[}\PY{n}{down\PYZus{}neighbor}\PY{p}{]} \PY{o}{=} \PY{n}{move\PYZus{}sequences}\PY{p}{[}\PY{n}{node\PYZus{}to\PYZus{}expand}\PY{p}{]} \PY{o}{+} \PY{l+s+s1}{\PYZsq{}}\PY{l+s+s1}{D}\PY{l+s+s1}{\PYZsq{}}
                    \PY{n}{neighbors}\PY{o}{.}\PY{n}{append}\PY{p}{(}\PY{n}{down\PYZus{}neighbor}\PY{p}{)}
                    
                \PY{k}{for} \PY{n}{neigh} \PY{o+ow}{in} \PY{n}{neighbors}\PY{p}{:}
                    \PY{k}{if} \PY{n}{neigh} \PY{o+ow}{not} \PY{o+ow}{in} \PY{n}{frontier} \PY{o+ow}{and} \PY{n}{neigh} \PY{o+ow}{not} \PY{o+ow}{in} \PY{n}{visited}\PY{p}{:}
                        \PY{n}{frontier}\PY{o}{.}\PY{n}{append}\PY{p}{(}\PY{n}{neigh}\PY{p}{)}
\end{Verbatim}


    \begin{Verbatim}[commandchars=\\\{\}]
{\color{incolor}In [{\color{incolor}6}]:} \PY{n}{solution} \PY{o}{=} \PY{n}{eight\PYZus{}puzzle\PYZus{}breadth\PYZus{}first\PYZus{}search}\PY{p}{(}\PY{n}{INITIAL\PYZus{}STATE}\PY{p}{,} \PY{n}{FINAL\PYZus{}STATE}\PY{p}{)}
\end{Verbatim}


    \begin{Verbatim}[commandchars=\\\{\}]
{\color{incolor}In [{\color{incolor}7}]:} \PY{n+nb}{print}\PY{p}{(}\PY{l+s+s1}{\PYZsq{}}\PY{l+s+s1}{Path to goal: }\PY{l+s+si}{\PYZob{}\PYZcb{}}\PY{l+s+s1}{\PYZsq{}}\PY{o}{.}\PY{n}{format}\PY{p}{(}\PY{n}{solution}\PY{o}{.}\PY{n}{path\PYZus{}to\PYZus{}goal}\PY{p}{)}\PY{p}{)}
        \PY{n+nb}{print}\PY{p}{(}\PY{l+s+s1}{\PYZsq{}}\PY{l+s+s1}{Cost of path: }\PY{l+s+si}{\PYZob{}\PYZcb{}}\PY{l+s+s1}{\PYZsq{}}\PY{o}{.}\PY{n}{format}\PY{p}{(}\PY{n}{solution}\PY{o}{.}\PY{n}{path\PYZus{}cost}\PY{p}{)}\PY{p}{)}
        \PY{n+nb}{print}\PY{p}{(}\PY{l+s+s1}{\PYZsq{}}\PY{l+s+s1}{\PYZsh{} visisted nodes: }\PY{l+s+si}{\PYZob{}\PYZcb{}}\PY{l+s+s1}{\PYZsq{}}\PY{o}{.}\PY{n}{format}\PY{p}{(}\PY{n}{solution}\PY{o}{.}\PY{n}{num\PYZus{}visited\PYZus{}nodes}\PY{p}{)}\PY{p}{)}
        \PY{n+nb}{print}\PY{p}{(}\PY{l+s+s1}{\PYZsq{}}\PY{l+s+s1}{Running time: }\PY{l+s+si}{\PYZob{}\PYZcb{}}\PY{l+s+s1}{\PYZsq{}}\PY{o}{.}\PY{n}{format}\PY{p}{(}\PY{n}{solution}\PY{o}{.}\PY{n}{running\PYZus{}time\PYZus{}secs}\PY{p}{)}\PY{p}{)}
        \PY{n+nb}{print}\PY{p}{(}\PY{l+s+s1}{\PYZsq{}}\PY{l+s+s1}{Memory used: }\PY{l+s+si}{\PYZob{}\PYZcb{}}\PY{l+s+s1}{\PYZsq{}}\PY{o}{.}\PY{n}{format}\PY{p}{(}\PY{n}{solution}\PY{o}{.}\PY{n}{used\PYZus{}memory}\PY{p}{)}\PY{p}{)}
\end{Verbatim}


    \begin{Verbatim}[commandchars=\\\{\}]
Path to goal: LURDRDLLURRDLLURRULLDRRULL
Cost of path: 26
\# visisted nodes: 164919
Running time: 221.6931688785553
Memory used: 1798776

    \end{Verbatim}


    % Add a bibliography block to the postdoc
    
    
    
    \end{document}
